\documentclass[letterpaper,titlepage,oneside]{article}
\usepackage{geometry}
\geometry{letterpaper, portrait, margin=1in}
\setlength\parindent{24pt}
\usepackage{booktabs}
\renewcommand{\arraystretch}{1.3}
\usepackage{listings}
\usepackage{graphicx}
\usepackage{color}
\usepackage{placeins}
\usepackage{amsmath, amsthm, amssymb}
\definecolor{dkgreen}{rgb}{0,0.6,0}
\definecolor{gray}{rgb}{0.5,0.5,0.5}
\definecolor{mauve}{rgb}{0.58,0,0.82}

\lstset{frame=tb,
  language=Java,
  aboveskip=3mm,
  belowskip=3mm,
  showstringspaces=false,
  columns=flexible,
  basicstyle={\small\ttfamily},
  numbers=none,
  numberstyle=\tiny\color{gray},
  keywordstyle=\color{blue},
  commentstyle=\color{dkgreen},
  stringstyle=\color{mauve},
  breaklines=true,
  breakatwhitespace=true,
  tabsize=4
}
\begin{document}
\pagenumbering{gobble}
\begin{center}
\section*{THE COOPER UNION \\FOR THE ADVANCEMENT OF SCIENCE AND ART\\[35pt]}
\section*{A Practical Method for the Implementation of \\Cascading Tetrominoes with Digital Logic\\[35pt]}
\subsection*{Arnold Wey, EE '18 \\[15pt]-\&-\\[15pt] Camilo Gaitan, CE '18\\[75pt]}
\subsection*{A thesis submitted in partial fulfillment of the requirements for the degree of Bachelor of Engineering\\[75pt]}
\subsubsection*{May 13, 2015\\[60pt]}
\subsubsection*{Professor Jared Harwayne-Gidansky\\[5pt]Thesis Advisor}
\end{center}
\clearpage
\pagenumbering{roman}

\section{TO DO}
Define stuff, like Tetrominoes
Introduction
Pictures ( Big, Labeled, Functional Blocks)
?State Diagrams?
Labelling of Pictures
PRNG? RNG? Investigate before using label
Make note of Abbreviations (Remove them)


\section*{Abstract}
The objective of the project is a digital logic port of the 1984 game Tetris, which involves the manipulation of a falling shape in a vertical board. Key features of gameplay are the random generation of falling Tetrominoes, the ability to move the falling shape (henceforth referred to as ``Mino'') left, right, and rotate it, and the deletion of full lines. 

The project demonstrates the basic functionality of Tetris. Additional safeguards could be implemented but weren't defined in the project specs, such as Left/Right collision detection, wall collision detection, increased falling speed as the game progresses, and scoring system.
\clearpage

\section*{Acknowledgements}
We cannot express enough thanks to the Electrical Engineering Faculty for their continued support and 
encouragement:  Professor Fred Fontaine, Electrical Engineering Chair; Professor Toby Cumberbatch; Professor Sam Keene; Professor Carl Sable; and Professor Stuart Kirtman, along with Dino Melendez for being endless sources of knowledge and enlightenment. We offer our sincere appreciation for the learning opportunities and facilities provided by The Cooper Union. 

Our completion of this project could not have been accomplished without the advice of the Electrical Engineering upperclassmen: Chris Curro, Howie Chen, Stephen Leone, Justin Alexander, Neema Aggarwal, and Venkat Kuruturi. Their help with WinCUPL, oscilloscopes, .bin file manipulation, and debugging was essential. 

To Yuliya Koshkina –thank you for allowing me time away from you to put wires into holes. [Insert Romantic Shit, Camilo]  

Thanks to our parents as well, Gwodonq Wey, Bihyueh Chen Wey, [Camilo's mam\'a y pap\'a]. The opportunities you gave us through your own sacrifices are boundless, along with our appreciation for everything you still do.

Finally, to our sarcastic, tyrannical, tough-love professor, Jared Harwayne-Gidansky: our deepest gratitude. Your encouragement when the times got rough are much appreciated and duly noted. It was a great comfort and relief to know that you cared for our well-being and learning experience. \\[5pt]
You have our heartfelt thanks.
\clearpage
\tableofcontents
\pagebreak
\pagenumbering{arabic}
\section{Introduction}

\clearpage
\section{Design Considerations}

\subsection{Overview}
Tetrominoes are hard coded into an EEPROM, which is addressed by counters controlled by buttons. Random shapes are generated by a 3-bit RNG.

The Field is rendered row-by-row from the bottom up, while processing each row and checking for a full row, an empty row, and selecting either the current row or the next row to write into RAM. The bottom row of the mino, RowBottom, corresponds to its location in the board and is compared with the current row being processed. This comparison, carried out by Overlap, determines whether to enable the output of the corresponding row from the EEPROM. RowBottom is maintained by a 3-bit down counter.

The output is displayed by sinking current on 1 row of the LED matrix at a time, while sourcing power to the appropriate "x" ordinates. This happens at a high frequency; the frame looks whole via persistence of vision.

\subsection{Trimming}
These solutions were devised through an iterative process which slowly decreased the number of boards required to implement Tetris. The original design required over 30 boards, which has slowly been trimmed down to around 14.

Tetris has been traditionally implemented on a 10x20 board, which was an awkward number of inputs to MUX, and required an extra chip for every part of the circuit that used MUXs, as well as 3 4-bit shift registers instead of 2. 

This was compounded by the original implementation to display the board, where the Tetromino would be loaded into a RAM separate from the RAM storing the FIELD. The outputs from the two RAMs would be time-	domain MUXed to individual rows of LEDs. 

Then, we considered loading each shape into 4 8-bit Universal Shift Registers (8 4-bit SRs), which could be manipulated and fed into rotation matrices in order to manipulate the output. While building, we realized that hard-coding the possible states of each row containing a Tetromino could fit inside the EEPROM, completely eliminating the need for the 4 shift registers storing each row that would handle shifting left and right, not to mention save the cost of Universal Shift Registers, which are quite costly. \\
This brought the total board count to around 18, but required writing 1024 unique rows

The need to manually write 1024 bytes prompted an investigation in automating the process. There's no simple way to handle rotation, so 
$ 4 \,rows\, *\, 8\, shapes\, *\, 4 \,orientations \,=\, 128\, rows$
were written manually into input.txt. "shapeTest.c" prints out all the possible offsets of each shape's orientations, read from input.txt. The stdout from shapeTest is redirected into Swag.txt. The debugging output is manually cleaned up, and used as input to bintext2bin.\\
bintext2bin.c converts input from a text file containing 8bit rows encoded as ASCII 1's and 0's into a bin file with equivalent information. (output.bin)\\
This bin file was loaded into the memory buffer at even addresses, for pinning convenience. Additionally, this allows the LSB on the EEPROM to be used as a logical disable.

Asserting 1 row at a time from the EEPROM and the RAM allows us to do away with another set of MUXs, and performing a bit-wise OR to control the display.

After cutting down the number of boards required to control the shape, we needed to reexamine the logic handling the RAM.

In order to access and manipulate information within the RAM, they must be stored and displayed on Shift Registers. In addition the RAM needs to be addressed to both read and write. This sequential nature made the 4017-Decade Counter an obvious choice.

A lot of sequential logic processing requires enabling different circuits at the same frame in a timeline, depending on different conditions. This is accomplished through the use of "flags," which are flip flops which store a certain condition (Full Row, Empty Row, etc.)

The use of flags greatly simplified shifting different rows and selecting between inputs to MUXs in general. For example, the outputs from Shift\_Register\_Two could go to an 8 input NOR, which is HIGH when Shift\_Register\_Two is empty. When this Flip Flop is high,the Shift-Down MUX will select the output from Shift\_Register\_One, which stores the "next row" in the RAM. 

Writing back into the RAM requires having read and write information asserted on the same bus. This requires the use of a Tri-State enabled MUX.

A few more boards were saved by implementing Truth Table~\ref{table:OverLap_Unminimized} on page~\pageref{table:OverLap_Unminimized}
 into a GAL chip, Overlap.
 
That Truth Table simplifies into Truth Table~\ref{table:Overlap_Minimized} on page~\pageref{table:Overlap_Minimized}, and results in Equation /eqref{}: 
This had few enough product terms to fit onto a GAL16v8, and upon securing approval, we programmed the chip using WinCUPL and the ChipMaster 6000 graciously provided to us by the Cooper Union Electrical Engineering Department.

The Overlap.PLD source code, followed by snippets of the test vector file, can be found on page~\pageref{code:Overlap}.

\clearpage

\section{Final Implementation}
\textit{Note: the following discussion will include C-Style pointer notation, where *address refers to the value stored at the given address, and \&value refers to the address where value resides.}

The current "State" is stored in a shift register at the start of each row processing cycle. This state represents the address of the RAM which will be written into. 1st, the shift registers are clocked in such a way as to store *State and *(State+1) into two shift registers. The exact order of clocking is included on page~\pageref{table:Clock}.

[WIP]

\section{Equations}
\subsection{Logic Equations}

\subsubsection{Overlap}

\begin{align*} 
Overlap\,=\, 	&(B_{2}+B_{3}+B_{4}+!A_{2})*(B_{2}+B_{3}+!A_{2}+!A_{4})*(B_{2}+B_{3}+!A_{2}+!A_{3})*\\
			&(B_{2}+B_{4}+!A_{2}+!A_{3})*(B_{2}+!A_{2}+!A_{3}+!A_{4})*(!B_{1}+A_{1})*(!B_{2}+!B_{4}+!A_{2}+A_{3}+A_{4})*\\
			&(B_{1}+!A_{1}+!A_{2})*(!B_{2}+A_{1}+A_{2})*(B_{1}+B_{2}+!A_{1})*(B_{2}+!B_{3}+A_{2}+A_{3})*\\
&(!B_{2}+B_{3}+B_{4}+A_{2})*(!B_{2}+A_{2}+!A_{3}+!A_{4})*(!B_{2}+!B_{3}+!A_{2}+A_{3})*\\
&(B_{2}+!B_{3}+!B_{4}+A_{2}+A_{4})*(B_{2}+!B_{4}+A_{2}+A_{3}+A_{4})*(!B_{2}+B_{3}+A_{2}+!A_{3})*\\
&(!B_{2}+B_{3}+A_{2}+!A_{4})*(!B_{2}+B_{4}+A_{2}+!A_{3})*(!B_{2}+!B_{3}+!B_{4}+!A_{2}+A_{4})*\\
\end{align*}


\subsubsection{4 Bit Select}
\begin{equation}\nonumber
Y_{n} = (A_{n} \,\&\, !ADD) \,+\, (B_{n} \,\&\, ADD)
\end{equation}

\subsubsection{8AndOr}
\begin{equation}\nonumber
AND \,=\, \sum_{i=0}^{7} I_{i}
\end{equation}

\begin{equation}\nonumber
OR \, = \, \prod_{i=0}^{7} I_{i}
\end{equation}

\subsection{Calculations}
\begin{align*}
Min\,Frequency_{Persistance\, Of\, Vision} &\equiv& 50\,Hz\\
50 \frac{Hz}{row} * 16 rows&=& 800\,Hz \\
F_{min\,Timer} * \frac{1}{10} &=& \,800Hz \\
F_{timer} &>& \,8kHz \\
\frac{1.44}{C(R_{a} + 2 R_{b})} &>& 8\,kHz \\
\end{align*}

\subsubsection*{Decade Timer Frequency Calculations}
\begin{align*}
\frac{1.44}{C(R_{a} + 2 R_{b})} &=& Frequency \\
R_{a} &=& 550\,\Omega\\ 
R_{b} &=& 810\,\Omega\\
C &=& .015\,\mu F\\
Frequency &=& 442\,kHz\\
\end{align*}
\begin{center}
\begin{small}
\textit{Note: The 555 is producing  $\sim 38.5\,kHz$, as measured by an oscilloscope.}
\end{small}
\end{center}

\subsubsection*{Row Falling Frequency Calculations}
\begin{align*}
Frequency_{fall} &=& \frac{1}{3} \, Hz\\
\frac{1.44}{C(R_{a} + 2 R_{b})} &=& Frequency \\
R_{a} &=& 470\,k\Omega\\ 
R_{b} &=& 470\,k \Omega\\
C &=& 3.3\, F\\
Frequency &=& 0.310\,Hz\\
\end{align*}
\begin{center}
\begin{small}
\textit{Note: The 555 is producing $ \sim 0.33\,kHz$, as measured by an oscilloscope.}
\end{small}
\end{center}

\clearpage
\section{Sample Code}
\subsection{WinCUPL}
\paragraph*{Overlap.PLD\\}
\label{code:Overlap}
\begin{lstlisting}
Name 	Overlap;
Partno 	01;
Date 	4/24/2015;
Rev 	01;
Designer 	Arnold Wey;
Company 	CU Later;
Assembly 	None;
Location 	None;
Device 	g16v8;

/**Inputs**/
Pin 1 = A1;
Pin 2 = A2;
Pin 3 = A3;
Pin 4 = A4;
Pin 5 = B1;
Pin 6 = B2;
Pin 7 = B3;
Pin 8 = B4;

/**Outputs**/
Pin 15 = I1;
Pin 16 = I2;
Pin 17 = I3;
Pin 18 = I4;
Pin 14 = Overlap;
Pin 13 = NotOverlap;

O0 = B2   # B3    # B4    # !A2 ;
O1 = B2   # B3    # !A2   # !A4 ;
O2 = B2   # B3    # !A2   # !A3 ;
O3 = B2   # B4    # !A2   # !A3 ;
O4 = B2   # !A2   # !A3   # !A4 ;
O5 = !B1  # A1  ;
O6 = B1   # !A1   # !A2 ;
O7 = !B2  # A1    # A2  ;
O8 = B1   # B2    # !A1 ;
O9 = B2   # !B3   # A2    # A3  ;
O10 = !B2 # B3    # B4    # A2  ;
O11 = !B2 # A2    # !A3   # !A4 ;
O12 = !B2 # !B3   # !A2   # A3  ;
O13 = B2  # !B3   # !B4   # A2    # A4  ;
O14 = B2  # !B4   # A2    # A3    # A4  ;
O15 = !B2 # B3    # A2    # !A3 ;
O16 = !B2 # B3    # A2    # !A4 ;
O17 = !B2 # B4    # A2    # !A3 ;
O18 = !B2 # !B3   # !B4   # !A2   # A4  ;
O19 = !B2 # !B4   # !A2   # A3    # A4  ;

/*Combining Terms*/
I1 = [O0, O1, O2, O3, O4, O10,O15]:&;
I2 = [O5, O6, O7, O8, O9]:&;
I3 = [O11,O13]:&;
I4 = [O16,O17,O18,O19]:&;

/*Final Terms*/
Overlap = [I1, I2, I3, I4,O14,O12]:&;
NotOverlap = !Overlap;



\end{lstlisting}

\paragraph*{Overlap.SI\\}
\label{code:OverlapSi}
\begin{lstlisting}

Name     	Overlap;
PartNo   	01;
Date     	4/24/2015;
Revision 	01;
Designer 	Arnold Wey;
Company  	CU Later;
Assembly 	None;
Location 	None;
Device   	g16v8;

ORDER: B1, B2, B3, B4, A1, A2, A3, A4, Overlap, NotOverlap; 

VECTORS:
00000000HL
00000001HL
00000010HL
00000011HL
00000100LH
00000101LH
00000110LH
00000111LH
00001000LH
   ...

\end{lstlisting}

\pagebreak
\paragraph*{8AndOr.PLD\\}
\label{code: 8AndOr}
\begin{lstlisting}

Name 8AndOr;
Partno 01;
Date 4/15/2015;
Rev 01;
Designer Arnold Wey;
Company CU Later;
Assembly None;
Location None;
Device g16v8;
/**Inputs**/

Pin 5 = I7;
Pin 6 = I6;
Pin 7 = I5;
Pin 8 = I4;
Pin 9 = I3;
Pin 11 = I0;
Pin 12 = I1;
Pin 13 = I2;

/**Outputs**/
Pin 14 = O4;
Pin 15 = O5;
Pin 16 = O6;
Pin 17 = O7;
Pin 18 = OR;
Pin 19 = AND;
AND = [I7, I6, I5, I4, I3, I0, I1, I2]:&;
OR = [I7, I6, I5, I4, I3, I0, I1, I2]:#;
O4 = I4;
O5 = I5;
O6 = I6;
O7 = I7;


\end{lstlisting}

\pagebreak
\paragraph*{8AndNor.SI}
\label{code: 8AndNorSi}
\begin{lstlisting}
Name     8AndNor;
PartNo   01;
Date     4/15/2015;
<Revision></Revision>;
Designer Arnold Wey;
Company  CU Later;
Assembly None;
Location None;
Device   g16v8;


ORDER: I7, I6, I5, I4, I3, I0, I1, I2, AND, OR; 


VECTORS:
00000000LL
00000001LH
00000010LH
00000011LH
00000100LH
00000101LH
00000110LH
00000111LH
00001000LH
00001001LH
00001010LH
00001011LH
00001100LH
   ...
\end{lstlisting}

\pagebreak
\paragraph*{4BSel.PLD}
\label{code: 4BSel}
\begin{lstlisting}
Name 4bSel;
Partno 01;
Date 4/15/2015;
Rev 01;
Designer Arnold Wey;
Company CU Later;
Assembly None;
Location None;
Device g22v10;
/**Inputs**/
Pin 1 = ADD;

Pin 4 = A0;
Pin 5 = A1;
Pin 6 = A2;
Pin 7 = A3;
Pin 8 = B0;
Pin 9 = B1;
Pin 10 = B2;
Pin 11 = B3;
/**Outputs**/

Pin 18 = Y3;
Pin 19 = Y2;
Pin 20 = Y1;
Pin 21 = Y0;

Y0 = A0 & !ADD;
APPEND Y0 = B0 & ADD;
Y1 = A1 & !ADD;
APPEND Y1 = B1 & ADD;
Y2 = A2 & !ADD;
APPEND Y2 = B2 & ADD;
Y3 = A3 & !ADD;
APPEND Y3 = B3 & ADD;

\end{lstlisting}
\pagebreak
\paragraph*{4BSel.SI}
\label{code: 4BSelSi}
\begin{lstlisting}
Name     4bSel;
PartNo   01;
Date     4/15/2015;
<Revision></Revision>;
Designer Arnold Wey;
Company  CU Later;
Assembly None;
Location None;
Device   g22v10;


ORDER: A0, ADD, B0, A1, B1, A2, B2, A3, B3, Y0, Y1, Y2, Y3; 


VECTORS:
000000000LLLL
000000001LLLL
000000010LLLH
000000011LLLH
000000100LLLL
000000101LLLL
000000110LLLH
000000111LLLH
000001000LLHL
000001001LLHL
000001010LLHH
000001011LLHH
000001100LLHL
000001101LLHL
000001110LLHH
000001111LLHH
\end{lstlisting}

\clearpage

\subsection{regEx}
\paragraph*{Bill Of Materials RegEx\\}
\label{code: RegexClean}
\begin{lstlisting}
"([0-9 a-z A-Z \-]{0,30})";"(SO|DIL)([a-z 0-9 A-Z \/ ]{0,50})";"([0-9 a-z A-Z \- \/ \, \_]{0,1000})";
\end{lstlisting}

\clearpage

\subsection{C Source}
\paragraph*{skewGen.c\\}
\label{code: skewGen.c}
\begin{lstlisting}
#include <stdio.h>
#include <stdbool.h>
#include <stdlib.h>
#include <string.h>

#include "tetris.h"
#include "BoolStr.h"

#define BUFSIZE 256

main() {
	char	buf[BUFSIZE];
	int	uc;
	int	i, len, linenum, numwritten, numblank;
	FILE	*fpin, *fpout;
	const char	*infile, *outfile;

	infile = "input.txt";	// Name of the input (text) file
	outfile = "output.txt"; // Name out the output (text) file
	
	// Open in text mode
	if( (fpin = fopen(infile, "r")) == NULL ){ 
		printf("Cannot open input file '%s'\n", infile);
		exit(1);
	}

	if( (fpout = fopen(outfile, "wb")) == NULL ){ // Open in binary mode
		printf("Cannot open output file '%s'\n", outfile);
		exit(1);
	}

	shape swag[7];
	swag[0].name = 'I';
	swag[1].name = 'J';
	swag[2].name = 'L';
	swag[3].name = 'Z';
	swag[4].name = 'S';
	swag[5].name = 'T';
	swag[6].name = 'O';

	for (int i = 0; i < 7; ++i){
		swag[i].orientation = 4;
		swag[i].row = 4;
		swag[i].col = 8;
	}

	linenum = 0;
	numwritten = 0;
	numblank = 0;
	int sha = 0;
	int ori = 0;
	int row = 0;

	while( fgets(buf, BUFSIZE, fpin) != NULL ){
		linenum++;
		// If last char is not newline then may not have full line
		len = strlen(buf);
		if( buf[len - 1] != '\n' ){
			printf("Did not read full line at LINE #%d ('%s')\n",
				linenum, buf);
			exit(1);
		}
		buf[len - 1] = '\0';	// Get rid of newline
		len--;			// Adjust length

		// Skip blank lines
		if( len == 0 ){	
			printf("Skipping blank line at LINE #%d\n", linenum);
			numblank++;
			continue;
		}

		// Make sure number of chars in line is correct
		if( len != 8 ){
			printf("Line wrong length at LINE #%d ('%s')\n",
				linenum, buf);
			exit(1);
		}

		//convert buf into array of bools
		for (int i = 0; i < len; ++i){
			if (buf[i] == '0')
				swag[sha].up[ori][row][i] = 0;
			else if (buf[i] == '1')
				swag[sha].up[ori][row][i] = 1;
			else
				printf("Invalid arguement at LINE #%d ('%s')\n", linenum, buf );
		}
	
		//Stick buf into array as array of bools in the right order
		row++;
		if (row > 3){
			row = 0;
			ori++;
		}
		if (ori >3){
			ori = 0;
			sha ++;
		}
		numwritten++;
	}

	// Close files
	fclose(fpout);
	fclose(fpin);
	
	printf("\nDone.\n");

	for (int sh = 0; sh < 7; ++sh){
		p3bs(swag[sh].up);
		for (int i = 0; i < 7; ++i){
			s3bs(swag[sh].up);
			p3bs(swag[sh].up);
		}
	}
	printf("number of rows written : %d\n" , numwritten);
}
\end{lstlisting}


\paragraph*{bintext2binary.c, Courtesy of Stuart Kirtman\\}
\label{code: bin2bin}
\begin{lstlisting}
// Program to convert lines of "binary" chars in text file to
// bytes in a binary file.
//
// SEK 4/13/2015
//

#include <stdio.h>
#include <stdlib.h>
#include <string.h>

#define BUFSIZE 256

int main()
{
	char	buf[BUFSIZE];
	int	uc;
	int	i, len, linenum, numwritten, numblank;
	FILE	*fpin, *fpout;
	const char	*infile, *outfile;

	// Change these as needed
	infile = "Swag.txt";	// Name of the input (text) file
	outfile = "output.bin"; // Name out the output (binary) file

	if( (fpin = fopen(infile, "r")) == NULL ) // Open in text mode
	{
		printf("Cannot open input file '%s'\n", infile);
		exit(1);
	}

	if( (fpout = fopen(outfile, "wb")) == NULL ) // Open in binary mode
	{
		printf("Cannot open output file '%s'\n", outfile);
		exit(1);
	}

	linenum = 0;
	numwritten = 0;
	numblank = 0;

	// Process each line in the input file
	while( fgets(buf, BUFSIZE, fpin) != NULL )
	{
		linenum++;

		// If last char is not newline then may not have full line
		len = strlen(buf);
		if( buf[len - 1] != '\n' )
		{
			printf("Did not read full line at LINE #%d ('%s')\n",
				linenum, buf);
			exit(1);
		}
		buf[len - 1] = '\0';	// Get rid of newline
		len--;			// Adjust length

		if( len == 0 )	// Skip blank lines
		{
			printf("Skipping blank line at LINE #%d\n",
				linenum);
			numblank++;
			continue;
		}

		if( len != 8 )	// Make sure number of chars in line is correct
		{
			printf("Line wrong length at LINE #%d ('%s')\n",
				linenum, buf);
			exit(1);
		}

		uc = 0;
		for(i=0; i < len; i++)
		{
			uc = uc << 1;
			if( buf[i] == '0' )	// 0 already shifted in
				;
			else if( buf[i] == '1' )
				uc |= 1;	// Make lsb 1
			else	// Error if char is not a '0' or a '1'
			{
				printf("Invalid character at LINE #%d ('%s')\n",
					linenum, buf);
				exit(1);
			}
		}

		fputc(uc, fpout);	// Write byte to (binary) output file
		numwritten++;

		// Inform user of progress
		printf("Line #%d: 'Writing '%s' as '%02X'\n", linenum, buf, uc);
	}

	// Close files
	fclose(fpout);
	fclose(fpin);

	printf("\n----------------------\n");
	printf("\nDone.\n");
	printf("Number of bytes written to '%s': %d\n", outfile, numwritten);
	printf("Number of blank lines: %d\n", numblank);

	exit(0);
}

\end{lstlisting}


\clearpage
\section{Tables}

\begin{table}[h!]
\begin{center}
\begin{tabular}{c|c|c|c|c|c|c|c|c|c|c|c}
\cline{2-11}

 & B1 & B2 & B3 & B4 & A1 & A2 & A3 & A4 & Overlap & !Overlap & \\ \cline{2-11}
 & 0 & 0 & 0 & 0 & 0 & 0 & 0 & 0 & H & L &   \\
 & 0 & 0 & 0 & 0 & 0 & 0 & 0 & 1 & H & L &   \\
 & 0 & 0 & 0 & 0 & 0 & 0 & 1 & 0 & H & L &   \\
 & 0 & 0 & 0 & 0 & 0 & 0 & 1 & 1 & H & L &   \\
 & 0 & 0 & 0 & 0 & 0 & 1 & 0 & 0 & L & H &   \\
 & 0 & 0 & 0 & 0 & 0 & 1 & 0 & 1 & L & H &   \\
 & 0 & 0 & 0 & 0 & 0 & 1 & 1 & 0 & L & H &   \\
 & 0 & 0 & 0 & 0 & 0 & 1 & 1 & 1 & L & H &   \\
 & 0 & 0 & 0 & 0 & 1 & 0 & 0 & 0 & L & H &   \\
 & 0 & 0 & 0 & 0 & 1 & 0 & 0 & 1 & L & H &   \\
 & 0 & 0 & 0 & 0 & 1 & 0 & 1 & 0 & L & H &   \\
 & 0 & 0 & 0 & 0 & 1 & 0 & 1 & 1 & L & H &   \\
 & 0 & 0 & 0 & 0 & 1 & 1 & 0 & 0 & L & H &   \\
 & 0 & 0 & 0 & 0 & 1 & 1 & 0 & 1 & L & H &   \\
 & 0 & 0 & 0 & 0 & 1 & 1 & 1 & 0 & L & H &   \\
 & 0 & 0 & 0 & 0 & 1 & 1 & 1 & 1 & L & H &   \\
 & 0 & 0 & 0 & 1 & 0 & 0 & 0 & 0 & L & H &   \\
 & 0 & 0 & 0 & 1 & 0 & 0 & 0 & 1 & H & L &   \\
 & 0 & 0 & 0 & 1 & 0 & 0 & 1 & 0 & H & L &   \\
 & 0 & 0 & 0 & 1 & 0 & 0 & 1 & 1 & H & L &   \\
 & 0 & 0 & 0 & 1 & 0 & 1 & 0 & 0 & H & L &   \\
 & 0 & 0 & 0 & 1 & 0 & 1 & 0 & 1 & L & H &   \\
 & 0 & 0 & 0 & 1 & 0 & 1 & 1 & 0 & L & H &   \\
 & \ldots & \ldots & \ldots & \ldots & \ldots & \ldots & \ldots & \ldots & \ldots & \ldots &  \\

\cline{2-11}
\end{tabular}
\caption{Overlap Unminimized, Abbreviated}\label{table:OverLap_Unminimized}

\end{center}
\end{table}

\begin{table}[h!]
\begin{center}

\begin{tabular}{c|c|c|c|c|c|c|c|c|c|c}
\cline{2-10}
 & A1 & A2 & A3 & A4 & B1 & B2 & B3 & B4 & Overlap &  \\ \cline{2-10}
 & 1 & 1 & 0 & 0 & 1 & 1 & X & X & 1 &  \\
 & 0 & 1 & 0 & 0 & 0 & 1 & X & X & 1 &  \\
 & 1 & 0 & 0 & 0 & 1 & 0 & X & X & 1 &  \\
 & 0 & 0 & 0 & 0 & 0 & 0 & X & X & 1 &  \\
 & 1 & 1 & 0 & X & 1 & 1 & 1 & X & 1 &  \\
 & 1 & 1 & X & 0 & 1 & 1 & 1 & X & 1 &  \\
 & 0 & 1 & X & 0 & 0 & 1 & 1 & X & 1 &  \\
 & 1 & 0 & 0 & X & 1 & 0 & 1 & X & 1 &  \\
 & 1 & 0 & X & 0 & 1 & 0 & 1 & X & 1 &  \\
 & 0 & 0 & 0 & X & 0 & 0 & 1 & X & 1 &  \\
 & 0 & 0 & X & 0 & 0 & 0 & 1 & X & 1 &  \\
 & 1 & 0 & 1 & X & 1 & 1 & 0 & X & 1 &  \\
 & 0 & 0 & 1 & X & 0 & 1 & 0 & X & 1 &  \\
 & 1 & 1 & 0 & X & 1 & 1 & X & 1 & 1 &  \\
 & 1 & 0 & 0 & X & 1 & 0 & X & 1 & 1 &  \\
 & 0 & 0 & 0 & X & 0 & 0 & X & 1 & 1 &  \\
 & 1 & 1 & X & X & 1 & 1 & 1 & 1 & 1 &  \\
 & 1 & 0 & X & X & 1 & 0 & 1 & 1 & 1 &  \\
 & 0 & X & 1 & 1 & 0 & 1 & X & 0 & 1 &  \\
 & 0 & 1 & 1 & 0 & 1 & 0 & 0 & X & 1 &  \\
 & 0 & X & 1 & 1 & 0 & 0 & 1 & 1 & 1 &  \\
 & 1 & 0 & 1 & 1 & 1 & 1 & X & 0 & 1 &  \\
 & 1 & 0 & X & 1 & 1 & 1 & 0 & 0 & 1 &  \\
 & 0 & 0 & X & 1 & 0 & 1 & 0 & 0 & 1 &  \\
 & 0 & 1 & 0 & 1 & 1 & 0 & 0 & 0 & 1 &  \\
 & 0 & 1 & 0 & X & 0 & 1 & 1 & X & 1 &  \\
 & 0 & 1 & X & 1 & 0 & 1 & 0 & 1 & 1 &  \\
\cline{2-10}
\end{tabular}

\caption{Overlap Minimized}\label{table:OverLap_Minimized}

\end{center}
\end{table}

\begin{table}[h!]
\begin{center}

\begin{tabular}{c|c|c|c|c|c|c|c|c|c|c|c|c|c|c}
\cline{2-14}
 & A0 & ADD & B0 & A1 & B1 & A2 & B2 & A3 & B3 & Y0 & Y1 & Y2 & Y3 &  \\ \cline{2-14}

 & 0 & 0 & 0 & 0 & 0 & 0 & 0 & 0 & 0 & L & L & L & L\\
 & 0 & 0 & 0 & 0 & 0 & 0 & 0 & 0 & 1 & L & L & L & L\\
 & 0 & 0 & 0 & 0 & 0 & 0 & 0 & 1 & 0 & L & L & L & H\\
 & 0 & 0 & 0 & 0 & 0 & 0 & 0 & 1 & 1 & L & L & L & H\\
 & 0 & 0 & 0 & 0 & 0 & 0 & 1 & 0 & 0 & L & L & L & L\\
 & 0 & 0 & 0 & 0 & 0 & 0 & 1 & 0 & 1 & L & L & L & L\\
 & 0 & 0 & 0 & 0 & 0 & 0 & 1 & 1 & 0 & L & L & L & H\\
 & 0 & 0 & 0 & 0 & 0 & 0 & 1 & 1 & 1 & L & L & L & H\\
 & 0 & 0 & 0 & 0 & 0 & 1 & 0 & 0 & 0 & L & L & H & L\\
 & 0 & 0 & 0 & 0 & 0 & 1 & 0 & 0 & 1 & L & L & H & L\\
 & 0 & 0 & 0 & 0 & 0 & 1 & 0 & 1 & 0 & L & L & H & H\\
 & \ldots{} & \ldots{} & \ldots{} & \ldots{} & \ldots{} & \ldots{} & %
\ldots{} & \ldots{} & \ldots{} & \ldots{} & \ldots{} & \ldots{} & \ldots{} &\\
\cline{2-14}
\end{tabular}
\caption{4BSel Unminimized, Abbreviated: }\label{table:4BSel_Unminimized}
\end{center}
\end{table}

\begin{table}[h!]
\begin{center}
\begin{tabular}{c|c|c|c|c|c|c|c|c|c|c|c|c}
\cline{2-11}
& I7 & I6 & I5 & I4 & I3 & I0 & I1 & I2 & AND & OR \\ 
\cline{2-11}
& 0 & 0 & 0 & 0 & 0 & 0 & 0 & 0 & L & L  \\  
& 0 & 0 & 0 & 0 & 0 & 0 & 0 & 1 & L & H \\
& 0 & 0 & 0 & 0 & 0 & 0 & 1 & 0 & L & H \\
& 0 & 0 & 0 & 0 & 0 & 0 & 1 & 1 & L & H \\
& 0 & 0 & 0 & 0 & 0 & 1 & 0 & 0 & L & H \\
& 0 & 0 & 0 & 0 & 0 & 1 & 0 & 1 & L & H \\
& \ldots{} & \ldots{} & \ldots{} & \ldots{} & \ldots{} & \ldots{} & %
\ldots{} & \ldots{} & \ldots{} & \ldots{}\\
& 1 & 1 & 1 & 1 & 1 & 1 & 1 & 0 & L & H \\
& 1 & 1 & 1 & 1 & 1 & 1 & 1 & 1 & H & H \\
\cline{2-11}
\end{tabular}
\caption{8 Input And \& Or, Abbreviated: }\label{8AndOr_Unminimized}
\end{center}
\end{table}

\begin{table}[h!]
\begin{center}
\begin{tabular}{c|c|c|c|c}
\cline{2-4}
&	Qty	&	Value	&	Description	&	\\
\cline{2-4}
& 1 & 4001D & Quad 2-input NOR  & \\
& 1 & 4002D & 4-input NOR & \\
& 1 & 4017D & COUNTER/DIVIDER & \\
& 5 & 4027D & Dual JK FLIP FLOP & \\
& 3 & 4029D & Binary/decimal up/down COUNTER  & \\
& 8 & 4035D & 4-bit parallel in/out SHIFT REGISTER  & \\
& 2 & 4048D & Expandable 8-input GATE & \\
& 10  & 4053D & Triple 2-channel ANALOG MULTIPLEXER & \\
& 1 & 4069D & Hex INVERTER  & \\
& 4 & 4071D & Quad 2-input OR & \\
& 9 & 4081D & Quad 2-input AND  & \\
& 3 & 4520N & Dual binary up COUNTER  & \\
& 2 & ATF16V8BS & CMOS PLD  & \\
& 1 & CY62256LL-SNC & 256K (32K x 8) CMOS-Static RAM  & \\
& 3 & LM555N  & TIMER & \\
& 1 & 2816  & MEMORY  & \\
\cline{2-4}

\end{tabular}
\end{center}
\caption{Bill Of Materials}\label{BOM}
\end{table}

\begin{table}[h!]

\begin{center}
\begin{tabular}{c|c}

\hline 
\rule[-1ex]{0pt}{2.5ex} Step & Action \\ 
\hline 
\rule[-1ex]{0pt}{2.5ex} 1 & $SR_{State}$ \\ 
\hline 
\rule[-1ex]{0pt}{2.5ex} 2 & $SR_{1}$ \\ 
\hline 
\rule[-1ex]{0pt}{2.5ex} 3 & $SR_{2}$ \\ 
\hline 
\rule[-1ex]{0pt}{2.5ex} 4 & $RAM Address$ \\ 
\hline 
\rule[-1ex]{0pt}{2.5ex} 5 & $SR_{1}$ \\ 
\hline 
\rule[-1ex]{0pt}{2.5ex} 6 & $SR_{Return}$ \\ 
\hline 
\rule[-1ex]{0pt}{2.5ex} 7 & Set $FF_{R/W}$ \\ 
\hline 
\rule[-1ex]{0pt}{2.5ex} 8 & Reset $FF_{R/W}$ \\ 
\hline 
\rule[-1ex]{0pt}{2.5ex} 9 & $Row_{Mino}$ \\ 
\hline 
\rule[-1ex]{0pt}{2.5ex} 10 & Set $FF_{Collision}$ \\ 
\hline 


\end{tabular} 
\end{center}\caption{Decade Counter Sequence}\label{table:Clock}
\end{table}


\clearpage
\section{Figures}
\begin{figure}[h!]
\begin{center}
\includegraphics[width = 0.8\textwidth]{"Circuit Diagrams/Collision Logic".png}
\caption{Collision Logic}
\label{fig:Collision}
\end{center}
\end{figure}

\begin{figure}[h!]
\begin{center}
\includegraphics[width = 0.8\textwidth]{"Circuit Diagrams/MinoControl".png}
\caption{Mino Control Logic}
\label{fig:Mino Control}
\end{center}
\end{figure}

\begin{figure}[h!]
\begin{center}
\includegraphics[angle = 90, width = \textwidth]{"Circuit Diagrams/Display".pdf}
\caption{Display Logic}
\label{fig:Display}
\end{center}
\end{figure}

\begin{figure}[h!]
\begin{center}
\includegraphics[angle = 90,  height = 20cm, keepaspectratio=true]{"FunctionalBlockParty".png}
\caption{Functional Block Diagram}
\label{fig:block}
\end{center}
\end{figure}

\clearpage




\end{document}


